%\section{Programming of differential equations}
The exercises of this chapter are about programming of solvers for ordinary differential
equations, corresponding to Appendix E in the book by Langtangen.


\begin{Problem}{\textbf{Solve a simple ODE with function-based code}}\label{simple_ode_func}

\noindent
This exercise aims to solve the ODE problem $u - 5u' = 0, u(0) = 0.1$, for $t \in [0,20]$.
\paragraph{a)} Identify the mathematical function $f(u,t)$ in the generic ODE form $u' = f(u,t)$.
\paragraph{b)} Implement the $f(u,t)$ function in a Python function.
\paragraph{c)} Use the \pythoninline{ForwardEuler} function from Section 2.1 of the
lecture notes \emph{Solving Ordinary Differential Equations in Python} to compute a numerical solu-
tion of the ODE problem. Use a time step $\Delta t = 5$.
\paragraph{d)} Plot the numerical solution and the exact solution $u(t) = 0.1e^{0.2t}$.
\paragraph{f)} Perform simulations for smaller $\Delta t$ values and demonstrate visually that the
numerical solution approaches the exact solution.

Filename: \texttt{simple\_ODE\_func.py}
\end{Problem}

\begin{Problem}{\textbf{Solve a simple ODE with class-based code}}\label{simple_ode_class}

\noindent
Solve the same ODE problem as in Exercise \ref{simple_ode_func}, but use the
\pythoninline{ForwardEuler} class described in Section 2.2 of the lecture
notes \emph{Solving Ordinary Differential Equations in Python}.
Implement the right-hand side function $f$ as a class too.

Filename: \texttt{simple\_ODE\_class.py}
\end{Problem}

\begin{Problem}{\textbf{Solve a simple ODE with the ODEsolver hierarchy}}

\noindent
Solve the same ODE problem as in Exercise \ref{simple_ode_func}, but use the
\pythoninline{ForwardEuler} class in the \pythoninline{ODESolver} hierarchy
from Section 2.4 of the lecture notes.

Filename: \texttt{simple\_ODE\_class\_ODESolver.py}
\end{Problem}


\begin{Problem}{\textbf{Decrease the length of the time steps}}
%\addcontentsline{toc}{section}{Exercise E.2: Decrease the length of the time steps - \texttt{decrease\_dt.py}}

\noindent We have the following differential equation
\begin{equation*}
\frac{dx}{dt} = \frac{\cos (6t)}{1 + t + x}.
\end{equation*}
Use Forward Euler to solve this differential equation numerically. You should solve it on the interval $t \in [0, 10]$, with initial condition $x(0) = 0$, and for number of time steps $n = \{20, 30, 35, 40, 50, 100, 1 000, 10 000\}$. Plot all the solutions in the same plot.

Filename: $\texttt{decrease\_dt.py}$
\end{Problem}

\begin{Problem}{\textbf{Implement the explicit midpoint method}}\label{Midpoint}
%\addcontentsline{toc}{section}{Exercise E.3: Implement Euler's midpoint method - \texttt{Midpoint.py}}

\noindent Implement a subclass \pythoninline{Midpoint} in the \pythoninline{ODESolver} hierarchy
from Section 2.4 of the lecture notes \emph{Solving Ordinary Differential Equations in Python}. The class
should implement the explicit midpoint method described in Section 2.3 in the lecture notes, defined
by
\begin{align*}
k_1 & = f(u_n, t_n), \\
k_2 & = f(u_n+\frac{\Delta t}{2} k_1, t_n+\frac{\Delta t}{2}), \\
u_{n+1} &= u_n + \Delta t\, k_2 .
\end{align*}
Test your implementation by solving $u' = f(u, t) = \cos(t)-t\sin(t), u(0) = 0$
and plot the numerical solutions obtained from both the explicit midpoint
method and Forward Euler together with the analytical solution
$u(t) = t\cos(t)$. Use $20$ time steps on the interval $t \in [0, 4\pi]$.

Filename: $\texttt{Midpoint.py}$
\end{Problem}

\begin{Problem}{\textbf{Implement Heun's method as a function}}\label{heuns_method_func}
\paragraph{a)}
\noindent
Heun's method is defined by
\begin{align*}
k_1 & = f(u_n, t_n), \\
k_2 & = f(u_n+\Delta t k_1, t_n+\Delta t), \\
u_{n+1} &= u_n + \Delta t\, (k_1/2 + k_2/2).
\end{align*}

Implement Heun's method using a plain function \pythoninline{heuns_method}
of the same type as the \pythoninline{ForwardEuler} function from Section 2.1
in the lecture notes \emph{Solving Ordinary Differential Equations in Python}.
Construct a test problem where you know the analytical solution.

\paragraph{b)}
Implement a test function \pythoninline{test_heuns_against_hand_calculations()}
where you compare your own hand calculated results with those returned from the
function. Compare the solution at one or two time steps.

\paragraph{c)}
Plot the difference between the numerical and analytical solution of your test
problem. You should demonstrate that the numerical solution approaches
the exact solution as $\Delta t$ decreases.

Filename: \texttt{heuns\_method\_func.py}
\end{Problem}


\begin{Problem}{\textbf{Solve an ODE describing cooling of coffee}}\\
\noindent
Newton's law of cooling,
\begin{equation*}
  \frac{dT}{dt} = -h(T-T_s)
\end{equation*}

can be used to see how the temperature T of an object changes because of heat
exchange with the surroundings, which have a temperature $T_s$. The
parameter $h$, with unit $s^{-1}$ is an experimentally determined constant
(heat transfer coefficient) describing how efficient the heat exchange with
the surroundings is. In this exercise we shall model the cooling of freshly
brewed coffee. First we must find a measure of $h$. Suppose we have measured
$T$ at $t_0=0$ and $t_1$. We can then use a Forward Euler approximation
of $\frac{dT}{dt}$ with one time step of length $t_1$,
\begin{equation*}
   \frac{T(t_1)-T(0)}{t_1} = -h(T(0)-T_s)
\end{equation*}
to make the estimate
\begin{equation*}
   h = \frac{T(t_1)-T(0)}{t_1(T_s-T(0))}.
\end{equation*}

\paragraph{a)}
Make a \pythoninline{class Cooling} containing the
parameters $h$ and $T_s$ as data attributes. Let these parameters
be set in the constructor. Implement the right-hand side of the
ODE in a \pythoninline{__call__(self, T, t)} method. We will use a
class from the \pythoninline{ODESolver} described in Section 2.4 of the
lecture notes to solve the ODE.

Create a stand-alone function \pythoninline{estimate_h(t1, Ts, T0, T1)} to
estimate the $h$ parameter based on the initial temperature and one data
point $(t_1, T(t_1))$. You can use this function to estimate a value
for $h$ prior to calling the constructor in the \pythoninline{Cooling} class.

\paragraph{b)}
Implement a test function \pythoninline{test_Cooling()} for testing
that class \pythoninline{Cooling} works. The test function should verify that the
\pythoninline{__call__} method returns the correct results for given values
of the arguments \pythoninline{T} and \pythoninline{t}.

\paragraph{c)} The temperature of freshly brewed coffee
is $95^{\circ}$ C at $t_0=0$ (when it is poured into your cup),
and $92^{\circ}$ C after 15 seconds, in a room with temperature $T_s$.

Solve the ODE numerically by a method of your choice from the \pythoninline{ODESolver}
hierarchy, for $T_s = 20$ and $T_s = 25$. Plot the two solutions in the same plot.
The time interval where you solve the equations should be chosen long enough
for the solutions to be "almost flat" and close to the respective room temperature.

Filename: \texttt{coffee.py}
\end{Problem}
\newpage

\begin{Problem}{\textbf{Compare numerical methods for solving ODEs}}
\paragraph{a)}Make two subclasses
\pythoninline{Midpoint} and \pythoninline{Heun} in the
\pythoninline{ODESolver} hierarchy described in Section 2.4 of the lecture notes
\emph{Solving Ordinarty Differential Equations in Python}. The classes should
implement the explicit midpoint method, defined in Problem \ref{Midpoint}, and Heun's method,
defined in Problem \ref{heuns_method_func}, respectively.

\paragraph{b)}
Test your implementation in the main block of your program, by solving the
ODE $u'(t) = t\cos(t)-\sin(t), u(0) = 2$. Compute the numerical solution using
Heun's method, the explicit midpoint method, and the 4th order Runge-Kutta method
from the \pythoninline{ODESolver} hierarchy.

Make one plot for each method, where the numerical method is compared to the
analytical solution $y(t) = t\sin(t)+2\cos(t)$. You should solve and plot on
the interval $t \in [0, 8\pi]$ using $n$ number of points on the interval,
for $n = \{20,25,50,150\}$. To create multiple plots you can either
use the \pythoninline{plt.subplot} function (assuming \pythoninline{mathplotlib.pyplot}
is imported as \pythoninline{plt}) to get multiple plots in the same window, or
you can use \pythoninline{plt.figure(1)}, \pythoninline{plt.figure(2)}, etc.,
to open multiple plot windows.

Remember to label/title each plot and to include a legend with the different values of $n$.

Filename: \texttt{compare\_methods.py}
\end{Problem}

\begin{Problem}{\textbf{Solving a system of ODEs; Lotka-Volterra model}}


  In Chapter \ref{ch:diff_eq} we solved a system of difference equations known
  as the Lotka-Volterra model, which describes the dynamic interactions between two
  species. The Lotka-Volterra model is more commonly formulated as a system of ODEs,
  which for two species $R$ and $F$ is given by
  \begin{align*}
  \frac{dR}{dt} &= aR - cRF, \\
  \frac{dR}{dt} &= ecRF - bF.
  \end{align*}
  Here $a$ is the natural growth rate of rabbits in the absence of predation, $b$ is
  the natural death rate of foxes in the absence of food (rabbits), $c$ is the death
  rate of rabbits due to predation and $e$ is the efficiency of turning
  predated rabbits into foxes.

  Solve this system using one of the solvers from the \texttt{ODESolver} hierarchy,
  up to time $T=500$ weeks. Use $a = 0.04$ weeks$^{-1}$, $b = 0.1$ weeks$^{-1}$,
  $c = 0.005$ weeks$^{-1}$, and $e = 0.2$ (dimensionless). Use initial conditions
  $R_0 = 100$ and $F_0 = 20$, and select a suitable time step $\Delta t$.

  Plot how the number of individuals in the populations
  vary with time.

Filename: \texttt{Lotka\_Volterra\_ODE.py}
\end{Problem}

\begin{Problem}{\textbf{Solving a system of ODEs; motions of a spring}}


The \pythoninline{ODESolver} hierarchy is adapted to cope with both scalar ODEs and systems of ODEs. In this exercise we will solve a system of ODEs using \pythoninline{ODESolver}.

Any ordinary differential equation of $n^{th}$ order can be written as a system of $1^{st}$ order equations.


We will see how a $2^{nd}$ order ODE can be used to study the motions of a spring. We have a block of mass $m$ hanging from a spring. The block is pulled downwards before it is released, causing the block to oscillate vertically. We will study the oscillation of the block in this exercise.


If you want to read the mathematical reasoning concerning this physical phenomena in detail, you may look up section 10.7 \textit{Svingninger og resonans} in Kalkulus (Lindstrøm, 2016). This is however not necessary to solve the exercise.

\paragraph{a)}
Consider the following $2^{nd}$ order ODE
\begin{equation*}
u'' + \frac{q}{m}u' + \frac{k}{m}u = 0,
\end{equation*}
which describes the motion of the spring. The initial condition is
\begin{equation*}
u_0 = \left(1, \ \ -\frac{q}{2m}+\sqrt{\frac{k}{m}-\left(\frac{q}{2m}\right)^2 }\right).
\end{equation*}
The parameter $k$ is a constant factor that describes the stiffness of the spring. The parameter $q$ describes the friction, for example air resistance. When $q=0$ there is no friction. In this entire exercise we will consider the case where $m=1$ and $k=2$.


Rewrite the equation to a system of ODEs by hand.
Then create a class \pythoninline{ProblemSpring} that takes the parameters $m$, $k$, and $q$ as instance attributes in the constructor. The vector $u_0$ should also be defined in the constructor, as it depends on $m$, $k$ and $q$. Create a special method \pythoninline{__call__(self, u, t)} that returns the vector $\left[\frac{d}{dt}u(t), \frac{d}{dt}u'(t) \right]$ that you calculated by hand.


\paragraph{b)}
Create a new class \pythoninline{SolverSpring} that takes \pythoninline{problem}, $T$ (time stop) and $n$ (time steps) as parameters in the constructor. The \pythoninline{problem} attribute is supposed to be an instance of the \pythoninline{ProblemSpring} class. Write a method \pythoninline{solve(self, method)} that solves our system of ODEs. This is an example of how it can be done:

\begin{python}
def solve(self, method=RungeKutta4):
        self.solver = method(self.problem)
        self.solver.set_initial_condition(self.problem.U0)
        time_points = linspace(0, self.T, self.n+1)
        U, self.t = self.solver.solve(time_points)
        self.u, self.u_der= U[:,0], U[:,1]
\end{python}
\noindent
 You can choose whether you want to use \pythoninline{RungeKutta4} or another method from \pythoninline{ODESolver}. Notice how we extract the inital condition \pythoninline{U0} from the \pythoninline{problem} instance of \pythoninline{class ProblemSpring}. Also note that the array \pythoninline{U} contains the values of the approximations to both $u(t)$ and $u'(t)$. Make sure you understand how to extract the two columns vectors \pythoninline{self.u} and \pythoninline{self.u_der}.

\paragraph{c)}
Write a method \pythoninline{exact(self)} that returns the analytical solution to our differential equation.
The analytic solution is given by:
$$
u(t) = \exp{-\frac{qt}{2m}}\left(\cos\left(t\sqrt{\frac{k}{m}-\left(\frac{q}{2m}\right)^2}\right) + \sin\left(t\sqrt{\frac{k}{m}-\left(\frac{q}{2m}\right)^2}\right)\right)
$$

\paragraph{d)}
Write a method \pythoninline{plot(self)} which plots the exact solution of $u(t)$ together with your approximations to $u(t)$ and $u'(t)$.

\paragraph{e)}
In the main block, create two problem instances of \pythoninline{ProblemSpring} that represents the two cases where $q=0$ and $q=0.3\sqrt{km}$, respectively.

Also, create a solver instance of \pythoninline{SolverSpring} for each of the two problem instances you just created. Let $T=30$ and $n=1000$.

Call the \pythoninline{solve} and the \pythoninline{plot} methods for both of your solver instances.

\paragraph{f)}
You will now create two test functions to ensure your implementation of the problem class and solver class is correct.

Make one test function where you compare the computed solution with the analytical solution of $u(t)$ for some given parameters, and let the test pass if the maximum error is less than some given tolerance.

Make another test function where you compare the computed derivative $u'(t)$ with the exact $u'(t)$ for the case where $m=1$, $k=2$ and $q=0$. In this case the analytical solution is
\begin{equation*}
u'(t)= \sqrt{2}\left(\cos\left(t\sqrt{2}\right) - \sin\left(t\sqrt(2) \right)\right).
\end{equation*}
Let the test pass if the maximum error is less than some given tolerance.

Filename: \pythoninline{spring_diffeq.py}
\end{Problem}

\begin{Problem}{\textbf{Modelling war between nations}}

\noindent We consider the interaction between two nations C1 and C2 and
a system of equations for modelling a conflict between these.  %\cite[p.396]{braun}.
Assuming that each nation is determined  to defend itself against a possible attack,
let $x(t)$ and $y(t)$ denote the armaments of
the first and second nation respectively. The change $x'(t)$ depends on the armaments of
$y(t)$. We assume that it's proportional to it, say $k y(t)$ for some positive constant
$k$. It also depends on the relationship of the two. Assuming anger
leads to increased armaments, let $g$ measure the relationship between them, positive
numbers meaning anger towards the other nation and 0 means neutral, and negative
numbers meaning disarmament.
The cost of having an army will restrain $x(t)$, represented by
a term $-\alpha x$ for some positive constant $\alpha$. Similar setup for $y(t)$ yields a
system of differential equations:
\begin{equation} \label{eq1}
    \frac{\dd x}{\dd t} = ky(t)-\alpha x(t) + g, \quad
    \frac{\dd y}{\dd t} = lx(t)-\beta y(t)+h.
\end{equation}
In the case where $x'(t)=y'(t)=0$ we have reached a stable point where neither
nation is increasing armies. We interpret such a fixed point as peace. In the case were $x(t)$
and $y(t)$ diverges we have an arms race, and we interpret this as war.

\paragraph{a)}
Make a function that solves the system (\ref{eq1}) with a numerical method
of your own choice (you may use ODESolver to do this) and a function that
plots the solution curves of $x(t)$ and $y(t)$ for given initial values.
Your program should not solve beyond the point where either $x$ or $y$ is zero.
We want to allow the value zero,
so have your program check whether $x$ and $y$ are larger than a very small negative number.
If you use ODESolver this can be done by defining a terminate function to send
with the solve method. Until otherwise specified, we let $t$ be the time measured in years.

Filename: $\texttt{C\_model.py}$

\paragraph{b)}
Modify your program to instead consist of two classes. The first class \pythoninline{ProblemConflict}
should contain the following:
\begin{itemize}
    \item An init method saving all information relevant to the problem (parameters etc)
    \item A call method such that the class can be called as a function. It should
    take an array $[x, y]$ of specific values of $x$ and $y$ at time $t$, the time $t$,
    and return the right hand side of the ODE system.
\end{itemize}
The second class \pythoninline{Solver} should consist of the following:
\begin{itemize}
    \item An init method that takes a problem instance on the form above, and a step length
    dt.
    \item A method that solves the problem, with the same restrictions as in Problem a).
            It should solve any problem on the same form as \pythoninline{ProblemConflict}
            that is given by two differential equations.
    \item A method that plots the solutions as in Problem a).
    \item A method that saves an image of the plot in \texttt{.png} format. When this is called,
    no plot should be visible to the user.
\end{itemize}
Use the parameter values $\alpha=\beta=0.2$, $g=h=0$, $x_0=10000$, $y_0=20000$.
Run the program once with $k=l=0.2$, and once for $k=l=0.3$ plotting the first 10 years.
What is the interpreted difference between these two?

\emph{Hint: You may need to convert step length to the number of time points to use.
This can be done as
}

\pythoninline{n = int(round( "Last time step" / "Step length")) + 1}

Filename: $\texttt{C\_model\_class.py}$

\paragraph{c)}
Let us consider the parameter values $k=l=0.9$, $\alpha=\beta=0.2$, and $g=h=0$.
One can argue that these give rough estimates for the arms race between 1909 to 1914
between the alliances of France and Russia, and Germany and Austria-Hungary. %\cite{braun}.
Assuming stability prior to this and negligible armies, we assume $x_0\approx0$ and $y_0\approx0$
(This does not mean that neither nation had armies, but that they were much smaller
prior to the arms race).
Solve the problem when $x_0$ and $y_0$ are zero versus when they are small positive numbers.
Plot the next 5 years of both in the same figure. What happens?

Filename: $\texttt{C\_model\_c.py}$

\paragraph{d)}
So far we have seen a model intended to describe a conflict situation prior to war.
The preceding model doesn't describe what happens during a war, but similar equations can.

First of all, we will work with two types of warfare. The conventional one, what we
know as regular warfare, and guerrilla warfare, where groups of combatants use military
tactics such as ambush, raids, hit-and-run, among others.

We first consider two conventional armies engaging. Let $x(t)$ and $y(t)$ denote
the respective forces (the number of soldiers) and $t$ denote the time measured
in days. The rate of change of $x(t)$ is affected by combat loss, operational loss
(non-combat related. e.g. disease, accidents), and reinforcements. Combat loss
should be proportional to the size of the opponent, represented by a term $-\alpha y(t)$, $\alpha > 0$.
The operational losses should depend only on $x(t)$, represented by a term $- k x(t)$, $k> 0$.
The reinforcements are represented by a function $f(t)$. In short-term warfare,
the operational losses are negligible, and we will assume it to be zero. We get
equations
\begin{equation*}
    \frac{\dd x}{\dd t} = -\alpha y(t) + f(t), \quad \frac{\dd y}{\dd t} = -\beta x + g(t).
\end{equation*}
For a conventional-guerrilla combat, $y(t)$ representing the guerrilla army,
we assume that the combat losses also depend on the size of its own army. As
guerrilla armies often use strategies of surprise and hidden attacks, it is safe
to assume that bigger losses are experienced when the army is larger. Let $-\beta x(t)y(t)$
denote the combat loss of a guerrilla army. By the same arguments as above, we get
equations
\begin{equation*}
    \frac{\dd x}{\dd t} = -\alpha y(t) + f(t), \quad \frac{\dd y}{\dd t} = -\beta x(t)y(t) + g(t).
\end{equation*}

Make two classes \pythoninline{ProblemCCWar} and \pythoninline{ProblemGCWar} on the same form as
\pythoninline{ProblemConflict}
representing the new problems. Note that $f$ and $g$ are now functions. To handle
the case of the provided $f$ and/or $g$ not being functions, you may need the commands \pythoninline{callable(f)}
which checks if $f$ a callable, and \newline \pythoninline{isinstance(f, (float, int))}
that checks if $f$ is a float or int, in order to convert a constant to a constant function.

Filename: $\texttt{CW\_model.py}$

\paragraph{e)}
The battle of Iwo Jima is a famous battle during World War II. It was fought on
an island just outside of Japan. America invaded the island on February 19, 1945,
and the fight lasted for 36 days. The Japanese army consisted of around 21500
soldiers, while the Americans had a number above 50000 by the 36th day.

During the war, the Japanese had no reinforcements. The Americans started with no
soldiers, but landed 54000 soldiers the first day, 6000 the third, 13000 the sixth, and none
for the remaining. The reinforcements is therefore given as
\begin{equation*}
    f(t)=\begin{cases}
        54000 & 0\leq t<1 \\
        0 & 1 \leq t < 2 \\
        6000 & 2 \leq t < 3 \\
        0 & 3 \leq t < 5 \\
        13000 & 5 \leq t < 6 \\
        0 & t \geq 6
    \end{cases}
\end{equation*}
It can be shown that good estimates for the parameter values are $\alpha=0.0544$
and $\beta = 0.0106$. %\cite{braun}.
The exact values on a day to day basis is
given in the file $\texttt{casualties.txt}$. Plot
the modeled American army vs the exact numbers, and $y(t)$ vs $x(t)$.
Both plots should have the x-axis corresponding to the first $T = 36$ days.

Filename: $\texttt{iwo\_jima.py}$

\paragraph{f)}
Find the least number of soldiers Japan would need (according to the model)
to have won the fight. You may round to the nearest hundred.
\emph{Hint: Check which army decreases to zero first. You might want to extend the
variable T for this.}

Filename: $\texttt{least\_number.py}$

\paragraph{g)}
Suppose the Japanese army was interpreted as a guerrilla army. Find a value for $\beta$
such that the fight is close. Is it likely that the outcome would be different if
America met a large guerrilla army?

Filename: $\texttt{guerrilla.py}$
\end{Problem}

\begin{Problem}{\textbf{Simulate the spreading of a disease}}\label{bjorgvin}

\noindent
In this exercise we will model epidemiological diseases by implementing the SIRD model. Suppose we have four categories of people: susceptible (S) who are healthy but may contract the disease, infected (I) who have developed the disease and can infect the susceptible population, recovered (R) who have recovered from the disease and become immune, and the deceased (D) who did not survive the disease. Let $S(t)$, $I(t)$, $R(t)$ and $D(t)$ be the number of people in category S, I, R and D, respectively at time $t$. We have that $S(t) + I(t) + R(t) + D(t) = N$, where $N$ is the size of the initial population. For simplicity, we will assume that the populations otherwise remains constant.

Normal interaction between infected and susceptible members of the population causes a fraction of the susceptible to contract the disease. The fraction of the susceptible population that becomes infected will depend on the likelihood of encountering an infected individual as well as how contagious the disease is. This will be proportional to the number of infected members of the population.

During a time interval $\Delta t$ starting at time $t$, the fraction of the susceptible population which contracts the disease is $\alpha I (t) \Delta t$. The number of people that moves from the S to the I category is given by
\begin{equation*}
    S(t+\Delta t) = S(t) - \alpha S(t)I(t)\Delta t.
\end{equation*}
Divide by $\Delta t$ and let $\Delta \rightarrow 0$ to get the differential equation:
\begin{equation} \label{eq:1}
S'(t)=-\alpha S(t) I(t).
\end{equation}

Per time unit a fraction $\beta$ of the infected will recover from the disease, and  a fraction $\gamma$ of the infected will die as a result of the disease. In a time $\Delta t$, $\beta I(t) \Delta t$ people of the infected population will recover and move from the I to the R category, and $\gamma I(t) \Delta t$ dies and move from the I to the D category. In the same time interval, $\alpha S(t)I(t)\Delta t$ from the S category will be infected and moved to the I category. The accounting for the I category therefore becomes
\begin{equation*}
I(t+\Delta t) = I(t) + \alpha S (t) I(t) \Delta t - \beta I(t)\Delta t - \gamma I(t) \Delta t,
\end{equation*}
which in the limit $\Delta t \rightarrow 0$ becomes the differential equation
\begin{equation} \label{eq:2}
I'(t) = \alpha S(t) I(t) - \beta I(t) - \gamma I(t) .
\end{equation}

The R category gets contributions from the I category:
\begin{equation*}
   R(t + \Delta t) = R(t) + \beta I(t) \Delta t,
\end{equation*}
and the corresponding ODE for $R$ reads
\begin{equation} \label{eq:4}
R'(t) = \beta I(t).
\end{equation}

Finally, the D category gets contributions from the I category as well:
\begin{equation*}
  D(t + \Delta t) = D(t) + \gamma I(t) \Delta t,
\end{equation*}
and the corresponding ODE for $D$ reads
\begin{equation} \label{eq:5}
D'(t) = \gamma I (t).
\end{equation}

The system (E.2)-(E.5) is what we will call a SIRD model.

\paragraph{a)}
Make three separate functions: (i) \pythoninline{SIRD(u,t)} which defines the right hand side of the ODE system of the SIRD model. The parameters $\alpha, \beta, \gamma$ can be defined as local variables inside the function. (ii) A function \pythoninline{solve_SIRD} for solving the system of differential equations in the SIRD model by a numerical method of your choice from the \pythoninline{ODESolver}. (iii) A function \pythoninline{plot_SIRD(u,t)} for visualising $S(t)$, $I(t)$, $R(t)$ and $D(t)$ in the same plot. Make sure to use labels in the plot.

\paragraph{}
We will use the functions from a) to study the spreading of one of the most devastating pandemics in human history, the Black Death. The Black Death, also called the Plague, was evidently spread to Norway in 1349 after a ship from England arrived in Bjørgvin (today Bergen), carrying the disease.

There lived approximately 7000 people in Bjørgvin in 1349. %\cite{berg}
Let's say the ship crew consisted of 30 men, which were all infected with the Plague. For simplicity we only consider human-to-human transmission of the disease (we do not consider the rats and fleas). We are interested in how the disease developed in Bjørgvin the first 8 weeks after the ship arrived.

We assume that the Plague was $90 \%$ deadly and that death usually occurred four days after being infected.

\paragraph{b)}
Adding the equations shows that $S'(t) + I'(t) + R'(t) + D'(t) = 0$, which means that $S(t) + I(t) + R(t) + D(t)$ must be constant. Perform a test at each time step by checking that $S(t) + I(t) + R(t) + D(t)$ equals $S(0)+I(0)+R(0)+D(0)$ within some small tolerance. Since \pythoninline{ODESolver} is used to solve the ODE system, the test should be implemented as a user-specified \pythoninline{terminate(u, t, k)} function that is called by the \pythoninline{solve} function at every time step. The \pythoninline{terminate} function should simply print an error message and return \pythoninline{True} for if $S+I+R+D$ is not constant.

\paragraph{c)}
Visualise first how the disease develops when $\alpha = 6.5 \cdot 10 ^{-5}$ and  print out the number of deceased after 63 days.

Certain precautions, like staying inside, will reduce $\alpha$. Try $\alpha = 5.5 \cdot 10 ^{-5}$ and  compare the plot with the plot where $\alpha = 6.5 \cdot 10 ^{-5}$. Comment how the  $\alpha$ influences $S(t)$.

Use the constants $\beta = 0.1/4$ and $\gamma = 0.9/4$ to describe the Plague in Bjørgvin. The initial conditions would be $S(0) = 7000$, $I(0) = 30$, $R(0) = 0$, $D(0) = 0$, $\Delta t = 1$, and $t \in [0,63]$. Time $t$ here is given in days.

\emph{As there do not exist any exact data from the condition in Norway during the Black Death, the parameters above are all fictional. However, there is a broad consensus that the disease killed more than half of Norway's population. Read more about the disease here: https://snl.no/svartedauden and https://sml.snl.no/svartedauden.}

Filename: \texttt{bjorgvin.py}
\end{Problem}


\begin{Problem}{\textbf{Introduce classes in the SIRD model}}\label{SIRD}

\noindent
Implement the SIRD model from exercise \ref{bjorgvin} in a module called \texttt{SIRD.py}. First we will create a class \pythoninline{Region} which can represent any region given the specific initial conditions. Then we will create a problem class \pythoninline{ProblemSIRD} and a solver class \pythoninline{SolverSIRD} to solve the SIRD system of differential equations for a given region.

\paragraph{a)}
Create a class \pythoninline{Region} which has three methods, a constructor, a method \pythoninline{set_SIRD_values(self, u, t)} and a method for plotting the SIRD values \pythoninline{plot(self, x_label)}.

The constructor should take in the name of the region and the initial conditions $S(0)$, $I(0)$, $R(0)$ and  $D(0)$. All five parameters should be stores as attributes in the class. You will also need an attribute \pythoninline{self.population} which is the total population of the region at $t_0$.

The method \pythoninline{set_SIRD_values(self, u, t)} should take out the SIRD values from \pythoninline{u} and store S, I, R, D and t as attributes of the class.

The method \pythoninline{plot(self, x_label)} should plot S, I, R and D in the same plot. The string for the \pythoninline{plt.xlabel} should be given as a parameter (as the units of time may vary), while the \pythoninline{plt.ylabel} should always be set to e.g "Population". Set the title of the plot to be the name of the region. Specify color and label for all the different SIRD categories (an example could be \newline \pythoninline{plt.plot(self.t, self.S, label='Susceptible', color='Blue')}). Do not include \pythoninline{plt.legend()} or \pythoninline{plt.show()} in the code. This is because we may later want to add labels to the graphs, and because will use this method to plot several subplots.

In a main block in the bottom of the file, create an instance of class \pythoninline{Region} called \pythoninline{bjorgvin} using the parameters found in Problem \ref{bjorgvin}.

\paragraph{b)}
 Create the class \pythoninline{ProblemSIRD}.  The constructor should take in the parameters $\alpha$, $\beta$, $\gamma$ and a region, which must be an instance of the class \pythoninline{Region}. The parameter $\alpha$ in the SIRD model can be constant or function of time. The implementation of \pythoninline{ProblemSIRD} should be such that $\alpha$ can be given as either a constant or a Python function. The constructor should therefore look like this:
\begin{python}
def __init__(self, region, alpha, beta, gamma):
    if isinstance(alpha, (float, int)): # number?
        self.alpha = lambda t: alpha    # wrap as function
    elif callable(alpha):
        self.alpha = alpha
    # Store the other parameters
    self.set_initial_condition()        # method call
\end{python}

Write the method \pythoninline{set_initial_condition(self)} which stores a list \newline \pythoninline{self.initial_condition} containing the initial values of $S(0)$, $I(0)$, $R(0)$, $D(0)$ (in this particular order). The initial values should be extracted from the class attribute \pythoninline{region}.

Write a method \pythoninline{get_population(self)} which simply returns the value of the population of the region, which is stored in the class attribute \pythoninline{region}.

Write a method \pythoninline{solution(self, u, t)} which simply calls the method \pythoninline{set_SIRD_values(u, t)} of the class attribute \pythoninline{region}.

Finally, write a special method \pythoninline{__call__(self, u, t)} which represents the right-hand side function of our SIRD system  of ODEs. This method will return a list of the calculated values of $S'(t), I'(t), R'(t)$ and $D'(t)$, in that order.


In the main block, create an instance of class \pythoninline{ProblemSIRD} called \pythoninline{problem} using the parameters found in Problem \ref{bjorgvin} and the \pythoninline{Region bjorgvin}.

\paragraph{c)}
Now we will create a class \pythoninline{SolverSIRD}. The class constructor should take the parameters \pythoninline{problem} (which must be an instance of class \pythoninline{ProblemSIRD}), $T$ (final time) and $\Delta t$, and store them as attributes. The constructor should also store  an attribute called \pythoninline{total_population}, which is obtained by calling the \pythoninline{get_population} method of \pythoninline{problem}.

Implement a method \pythoninline{terminate} which at each time step $t$ checks that $S(t)+I(t)+R(t)+D(t)$ equals \pythoninline{total_population} within some small tolerance. Simply print an error message and return \pythoninline{True} for termination if the total population is not constant. As the \pythoninline{ODESolver} will be used to solve the ODE system, this method will be called by the \pythoninline{solve} method in \pythoninline{ODESolver} at every time step.

Write a method \pythoninline{solve(self, method)} that solves the SIRD system of ODEs by a method of your choice from the \pythoninline{ODESolver}. Use the following sketch for this method:

\begin{python}
def solve(self, method=RungeKutta4):
    solver = method(self.problem)
    solver.set_initial_condition(...)
    t = np.linspace(...)
    # Remember to "activate" terminate in the solve call:
    u, t = solver.solve(t, self.terminate)
    # set the values of S, I, R, D, and t via
    # Problem class to the Region class:
    self.problem.solution(u, t)
\end{python}

In the main block, create an instance of class \pythoninline{SolverSIRD} called \pythoninline{solver} using the parameters found in Problem \ref{bjorgvin} and the \pythoninline{ProblemSIRD problem}. Plot the results by calling the \pythoninline{plot(x_label)} method of the \pythoninline{Region bjorgvin}. Label the graphs by calling \pythoninline{plt.legend()} before you call \pythoninline{plt.show()}.

Filename: \texttt{SIRD.py}
\end{Problem}


\begin{Problem}{\textbf{The SIRD model across regions}} \label{SIRD_interaction}

\noindent
The problem class from exercise \ref{SIRD} will only be able to model the spreading of a disease within one region. In this exercise we will improve our program with subclasses of \pythoninline{ProblemSIRD} and \pythoninline{Region} that permits  people in one region to get infected by people from another region. The likelihood of transmission of disease across regions  will depend on the distance between the regions.

We now denote the categories to specify which region they belong, such that e.g.  $S_i (t)$ would be the number of susceptible in the $i$-th region at time $t$.

Let $S'_i(t)$, $I'_i (t)$, $R'_i(t)$ and $D'_i(t)$ belong to the $i$-th region. In this new, interacting SIRD model the expressions of $R'_i(t)$ and $D'_i(t)$ are unchanged from the SIRD model explained in Problem \ref{bjorgvin}, such that
\begin{align*}
    R'_i (t) &= \beta I_i (t) \\
    D'_i (t) &= \gamma I_i (t) .
\end{align*}

The expression of  $I'_i (t)$ must consider the possibility of members of the population in the $i$-th region contracting the disease due to contact with another region. The expression of the $i$-th region is given by
\begin{equation*}
    S'_i = \sum_{j=1}^{M}-\alpha I_j e^{-d_{ij}}  S_i  ,
\end{equation*}
where $M$ is the number of regions and  $d_{ij}$ is the distance between the $i$-th and the $j$-th region. Note that the distance from a region to itself, $d_{ii},$ is always zero, which leaves the expression unchanged from the previous SIRD model.

The derivative for the infected category is then
\begin{equation*}
   I'_i (t) = -\beta I_i(t) - \gamma I_i (t) - S'_i (t) .
\end{equation*}

\paragraph{a)}
Create a subclass of \pythoninline{Region} called \pythoninline{RegionInteraction}. In addition to the parameters in \pythoninline{Region}, the \pythoninline{RegionInteraction} needs two parameters \pythoninline{latitude} ($\phi$)  and \pythoninline{longitude} ($\lambda$). The constructor should store the two values and convert them from degrees to radii, which is done by multiplying by $\frac{\pi}{180^{\circ}}$. Use the super class to store the rest of the parameters as attributes.

Create a method \pythoninline{distance(self, other)} which calculates the distance between the \pythoninline{self} region ($i$) and another region ($j$). The distance between two regions is given by the arc length $d$ between the regions:
\begin{equation*}
    d_{i j} = R_{Earth} \Delta \sigma_{i j} ,
\end{equation*}
where the radius of the Earth is $R_{Earth} = 64$ given in units of $10^4 \ \mathrm{m}$ and $\Delta \sigma$ is given by
\begin{equation*}
    \Delta \sigma_{i j} = \arccos \big(
        \sin \phi_i \sin \phi_j
        +  \cos \phi_i \cos \phi_j
        \cos \left(
                |\lambda_i - \lambda_j |
        \right)
    \big)  .
\end{equation*}
{\bf Warning:} Roundoff error may cause problems in the $\arccos$ function, since the arguments may become slightly $>1$ when a city is compared with itself, and this makes the function return a
NaN (Not a Number) value. To avoid this problem, add an if-test inside the distance function to
ensure that the argument to $\arccos$ is between 0 and 1.

\paragraph{b)}
Create a subclass \pythoninline{ProblemInteraction} of class \pythoninline{ProblemSIRD}. This class should take in a list of regions that must be instances of \pythoninline{RegionInteraction}. In adition to all the same parameters as its super class, the constructor of  \pythoninline{ProblemInteraction} should take in and store \pythoninline{region_name}. Send all other parameters to the constructor in the super class.

The method \pythoninline{get_population(self)} should store the total population of all the regions combined as \pythoninline{self.total_population}.

The method \pythoninline{set_initial_condition(self)} must create a (not nested) list  \pythoninline{self.initial_condition} with the initial values from all the regions. Loop over all the regions in the list \pythoninline{region} to get the list on the form
\begin{equation*}
    [ S_1(0), I_1(0), R_1(0), D_1(0), S_2(0), I_2(0), R_2(0), D_2(0), ..., D_M(0) ] .
\end{equation*}

The special method  \pythoninline{__call__(self, u, t)} should return a list with the derivatives at time $t$, in the same order as the list \pythoninline{self.initial_condition}. Below is a sketch of the implementation could look like:
\begin{python}
def __call__(self, u, t):
    n = len(self.region)
    # create a nested: SIRD_list[i] = [S_i, I_i, R_i, D_i]:
    SIRD_list = [u[i:i+4] for i in range(0, len(u), 4)]
    # crate a list containing all the I(t)-values:
    I_list = ...
    derivative = []
    for i in range(n):
        S, I, R, D = SIRD_list[i]
        dS = 0
        for j in range(n):
            I_other = I_list[j]
            dS += ...
        # calculate dI, dR and dD
        # put the values in the end of derivative
    return derivative
\end{python}

The method \pythoninline{solution} must provide the SIRD lists to all the regions. The example below shows how to create a nested list where  each element in the list contains the SIRD lists for a certain region. You do not have to use this code, but the result should be the same.
\begin{python}
def solution(self, u, t):
        n = len(t)
        n_reg = len(self.region)
        self.t = t
        self.S = np.zeros(n)
        self.I = ...
        SIRD_list = [u[:, i:i+4] for i in range(0, n_reg*4, 4)]
        for part, SIRD in zip(self.region, SIRD_list):
            part.set_SIRD_values(SIRD, t)
            self.S += ...
\end{python}
The attributes \pythoninline{self.S}, \pythoninline{self.I}, \pythoninline{self.R} and \pythoninline{self.D} should be the total values for all the regions combined, i.e., each SIRD-category summed over all regions.

Create a new method \pythoninline{plot(self, x_label)}. the method should create the same kind of plot as class \pythoninline{Region}'s method \pythoninline{plot(self, x_label)}, as explained in Problem \ref{SIRD}. The method in \pythoninline{ProblemInteraction} should plot the SIRD values for all the regions combined, and the title of the plot should be \pythoninline{self.region_name}.


Filename: \texttt{SIRD\_interaction.py}
\end{Problem}

\begin{Problem}{\textbf{Simulate the spreading of the Plague in Norway}}\label{plague}

\noindent
In this exercise we will use the classes \pythoninline{ProblemInteraction}, \pythoninline{SolverSIRD} and \pythoninline{RegionInteraction} from Problem \ref{SIRD_interaction} to simulate the spread of the Black Death in Norway.

We assume that the disease first broke out in Bjørgvin in 1349, and that this was the only case of the Plague in  country at that time. We assume that the disease ravaged for two years (104 weeks).

We divide Norway into five regions: Vestlandet, Sørlandet, Trøndelag, Østafjells and Nord-Norge. We assume that there lived about 370000 people in Norway; 90 000 in Vestlandet, 65 000 in Sørlandet, 80 000 in Østlandet, 70 000 in Trøndelag and 65 000 in Nord-Norge.

The positions of the regions are given by the latitude ang longitude of certain city in each region, which are given in the table below:
\begin{center}
\begin{tabular}{ |c|c|c|c| }
    \hline
 Region & city & latitude & longitude \\
 \hline
 Vestlandet & Bjørgvin: & 60\textdegree & 5.3\textdegree\\
 Sørlandet & Øyslebø: & 58\textdegree & 7.6\textdegree\\
 Østlandet & Brandbu: & 60\textdegree &  11\textdegree\\
 Trøndelag & Steinkjer: & 64\textdegree &  11\textdegree\\
 Nord-Norge & Bardufoss: & 69\textdegree & 19\textdegree\\
 \hline
\end{tabular}
\end{center}


\paragraph{a)}
Create a function for simulating the Plague in Norway. The function must contain one instance of class \pythoninline{RegionInteraction} for each region. Let $S(0)$ be the population in each region. Except for $I(0)$ = 30 in Vestlandet, let all the other initial conditions be set to zero.
Create a list of the five  \pythoninline{RegionInteraction} instances.


The function will plot one subplot of the disease progress of each region, and also one subplot for the total progress for all regions combined.
The function could look like this:
\begin{python}
def plague_Norway(alpha, beta, gamma, num_Weeks, dt):
    # create all five instances of regionInteraction
    # create a list containing all five regions
    # create problem, instance of ProblemInteraction
    # create solver, instance of SolverSIRD & call method solve
    plt.figure(figsize=(.., ..)) # set figsize
    index = 1
    # for each part in problem's attribute region:
        plt.subplot(2, 3, index)
        # Call plot method from current part
        index += 1
    plt.subplot(2, 3, index)
    # Call plot method from problem
    plt.legend()
    plt.show()
    # print the total percentage of deceased after two years
\end{python}
 \textit{Hint: Write the percent sign twice (\%\%) to get \% in a string.}

Call the function using the parameters $\alpha= 7 \cdot 10^{-6}$, $\beta=0.1/4$ and $\gamma=0.9/4$. The unit of time is weeks. Solve for $104$ weeks and set $\Delta t = 1/7$ such that one time step represent one day.

\paragraph{b)}
Until now we have assumed that $\alpha$ is constant. $\alpha$ is the parameter which describes the possibility that one susceptible meets an infected during the time interval $\Delta t$ with the result that the infected "successfully" infect the susceptible. In reality, this $\alpha$ may probably not be constant. In times of bad weather, more people would stay at home and $\alpha$ would be lower. Other factors could also decrease alpha, like better hygiene and wearing masks over a certain period of time. On the other hand; factors like nice weather, festivities and other reasons for people to gather would increase $\alpha$.
As the Plague ravaged in Norway so long ago it is hard to reproduce an accurate approximation of $\alpha$. Let us therefore assume that the weather and other factors made $\alpha$ look like this piecewise function:
\[
  \alpha(t) = \left\{
     \begin{array}{@{}l@{\thinspace}l}
      3 \cdot 10^{-5}  \hspace*{14mm}  0 \leq t \leq 2 \\
      6 \cdot 10^{-6} \hspace*{14mm} 4 < t \leq 19 \\
      6 \cdot 10^{-6} \hspace*{12.1mm} 24 < t \leq 41 \\
      7 \cdot 10^{-6}  \hspace{12.1mm} 49 < t \leq 75\\
      1 \cdot 10^{-6} \hspace{18mm} \text{else}
     \end{array}
   \right.
\]
Implement $\alpha(t)$ as a piecewise function \pythoninline{alpha(t)}. Call \pythoninline{plague_Norway} using the piecewise function for $\alpha (t)$. Keep the rest of the values from part a).

You may notice that by this model, the Plague barely reached Nord-Norge at all. In fact, we aren't even sure today whether the Plague ravaged in the Northern part of Norway or not.

\paragraph{c)}

In the Norwegian legends, Pesta was the literal personification of the Plague. Pesta was depicted as an old woman who wandered the countryside, carrying either a rake or a broom. Legend has it that if she passed by your home while carrying a rake,  someone in the household would be infected by the plague while the rest would be spared. However, if she was carrying a broom then there were not much hope for anyone.

Implement a function \pythoninline{pesta(t)} where you generate a random number between 0 and 20. You can use the \pythoninline{randint} function imported from the \pythoninline{random} module like this:
\begin{python}
from random import randint
number = randint(0,20)
\end{python}
If the generated number is 13, Pesta is at your door carrying a broom. The function should return ten times the value of \pythoninline{alpha(t)} from part b).
If the generated number is 4, then Pesta has brought her rake. The function should return five times the value of \pythoninline{alpha(t)}.
Otherwise, Pesta is not around today, and your function should return 0.4 times the value of \pythoninline{alpha(t)}.

Call \pythoninline{plague_Norway} using the \pythoninline{pesta(t)} function for the parameter $\alpha$. Keep the rest of the values from part a). In which region was Pesta most present?

Filename: \texttt{plague.py}

\end{Problem}
