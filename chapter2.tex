%\section{Loops and lists}

\begin{Problem}{\textbf{Multiply by five}} \label{prob21}
%\addcontentsline{toc}{section}{Exercise 2.1: Multiply by five   - \texttt{multiplication.py}}

\noindent Write a code printing out $5\cdot 1$, $5\cdot 2$, ..., $5\cdot 10$, using either
a \pythoninline{for} or a \pythoninline{while} loop.

Filename: $\texttt{multiplication.py}$
\end{Problem}


\begin{Problem}{\textbf{Multiplication table}}\label{prob22}
%\addcontentsline{toc}{section}{Exercise 2.2: Multiplication table - \texttt{mult\_table.py}}

\noindent Write a new code based on the one from Problem \ref{prob21}. This code should print the
whole miltiplication table from $1\cdot 1$ to $10 \cdot 10$.

\emph{Hint: You may want to consider using one loop inside another.}

Filename: $\texttt{mult\_table.py}$
\end{Problem}

\begin{Problem}{\textbf{Stirling's approximation}}\label{prob23}
%\addcontentsline{toc}{section}{Exercise 2.3: Stirling's approximation - \texttt{stirling.py}}

\noindent Stirling's approximation can be written $\ln (x!) \approx x\ln x - x$. This is a
good approximation for large $x$. Write out a nicely formatted table of integer
$x$ values, the actual value of $\ln (x!)$, and Stirling's approximation to $\ln (x!)$.

Filename: $\texttt{stirling.py}$
\end{Problem}


\begin{Problem}{\textbf{Errors in summation}}\label{sum_for}
%\addcontentsline{toc}{section}{Exercise 2.4: Errors in summation  - \texttt{sum\_for.py}}

\noindent The following code is supposed to compute the sum $s = \sum_{k = 1}^{M}\frac{1}{(2k)^2}$, for $M=3$.
\begin{python}
s = 0
M = 3

for i in range(M):
    s += 1/2*k**2

print(s)
\end{python}
The program has three errors and therefore does not work. Find the three errors and write a correct program. Put comments in your program to indicate what the mistakes were.

There are two basic ways to find errors in a program:
\begin{enumerate}
    \item Read the program carefully line by line, and think about the consequences of each statement. Look for inconsistencies in the form of variables being used before
		they are defined, and perform the calculations in the statements by hand to see how variables change their value.
    \item Run the program and check for errors. In a Python program there may be errors of two different types. The first is an error that Python itself will
		notice, and make the program stop with an error message. The message will indicate which line the error occurs, and alghough the error messages may be a little
		cryptic these errors are usually quite easy to find and correct. The second type are errors where the program seems to run and complete normally, but the
		result is not what we want. These errors are often harder to find, and a good method to find them is often to add \pythoninline{print} statements in the
		code to write intermediate values of variables to the screen, and compare these values with hand calculations.
\end{enumerate}
Try the first method first (\emph{code inspection}) and find as many errors as you can. Thereafter, try the second method, by for instance
adding a print statement inside the for-loop to print the value of \pythoninline{k} and \pythoninline{s}.

Filename: $\texttt{sum\_for.py}$
\end{Problem}


\begin{Problem}{\textbf{Sum as a \pythoninline{while} loop}}\label{sum_while}
%\addcontentsline{toc}{section}{Exercise 2.4: Errors in summation  - \texttt{sum\_for.py}}
Write the (corrected) \pythoninline{for} loop from the previous exercise as a
a \pythoninline{while} loop. Check that the two loops compute the same answer.

Filename: $\texttt{sum\_while.py}$
\end{Problem}

\newpage

\begin{Problem}{\textbf{Binomial coefficient}}\label{prob26}
%\addcontentsline{toc}{section}{Exercise 2.5: Binomial coefficient - \texttt{binomial.py}}

\noindent The binomial coefficient is indexed by two integers $n$ and $k$ and is written
$\binom{n}{k}$. It is given by the formula
\begin{equation}\label{eq: binomial_factorial}
\binom{n}{k} = \frac{n!}{k!(n-k)!}.
\end{equation}
We can write this out, and get
\begin{equation}\label{eq: binomial_product}
\binom{n}{k} = \prod_{j = 1}^{n - k}\frac{k + j}{j}.
\end{equation}

Use Eq.~(\ref{eq: binomial_product}) and a \pythoninline{for} loop to find the binomial coefficient
for $n = 14$ and $k = 3$. Compute the same value using Eq.~(\ref{eq: binomial_factorial})
and check that the results are correct.

\emph{Hint: The $\prod$ sign is a product sign. Thus
$\prod_{j = 1}^{n - k}\frac{k + j}{j} = \frac{k + 1}{1}\frac{k + 2}{2}
\dots\frac{k + (n - k)}{(n - k)}$. When checking the result you will need
\pythoninline{math.factorial}.}

Filename: $\texttt{binomial.py}$
\end{Problem}




\begin{Problem}{\textbf{Table showing population growth}} \label{population_table}

\noindent Consider again the bacterial colony from Problem \ref{prob12}. We now want to
study how the population grows with time, by calculating the number
of individuals for $n + 1$ uniformly spaced $t$ values throughout the interval
$[0, 48]$. Set $n = 12$ and write a \py{for} loop which computes and stores
$t$ and $N$ values in two lists \pythoninline{t} and \pythoninline{N}. Thereafter,
traverse the two lists with a separate \pythoninline{for} loop and
write out a nicely formatted table of $t$ and $N$
values.

Filename: \texttt{population\_table.py}
\end{Problem}

\begin{Problem}{\textbf{Nested list}}\label{population_table2}

\paragraph{a)} Compute two lists of $t$ and $N$ values as explained in Problem
\ref{population_table}. Store the two lists in a new nested list \pythoninline{tN1} such that
\pythoninline{tN1[0]} is the list containing $t$-values and \pythoninline{tN[1]} correspond to the list containing $N$-values. Write out a table
with $t$ and $N$ values in two columns by looping over the data in the \pythoninline{tN1}
list. Each $t$ and $N$ value should be written in the table as integers.

\paragraph{b)} Make a nested list \pythoninline{tN2} where \pythoninline{tN2[i]}
contains the $i$-th element of both the $t$-list and the $N$-list. Loop over the \pythoninline{tN2} list and write out the $t$ and $N$
values in the table as integers. The ouput should look exactly as in exercise
\textbf{a)}, although the underlying structure of the lists is different.

Filename: $\texttt{population\_table2.py}$
\end{Problem}

\begin{Problem}{\textbf{Calculate Cesaro mean}}

\noindent Let $(a_n)_{n=1}^\infty$ be a sequence of numbers, $s_k=\sum_{n=0}^k a_n=a_0+\dots,+a_k$,
and
\begin{equation*}
    S_N = \frac{1}{N-1}\sum_{k=0}^{N-1} s_k.
\end{equation*}
Let $(a_n)_{n=1}^\infty$ be the sequence with $a_n=(-1)^n$.
Calculate $S_N$ for \newline $N=1, 2, 3, 4, 5, 10, 50$ and print the results in a table.

Filename: $\texttt{cesaro\_mean.py}$
\end{Problem}

\begin{Problem}{\textbf{Catalan numbers}}

\noindent A number on the form
\begin{equation*}
    C_n=\frac{1}{n+1}\binom{2n}{n}=\frac{(2n)!}{(n+1)!n!}
\end{equation*}
is called a \emph{Catalan number}. Compute and print the first 10 Catalan numbers.

Filename: $\texttt{catalan.py}$
\end{Problem}

\begin{Problem}{\textbf{Molar Mass of Alkanes}}

\noindent
Alkanes are saturated hydrocarbons with the chemical formula $\mathrm{C_nH_{2n+2}}$. If there are $n$ Carbon atoms in the alkane, there will be $m = 2n+2$ Hydrogen atoms. The molar mass of the hydrocarbon is $M_{\mathrm{C_n H_{m}}} = n M_{\mathrm{C}} + m M_{\mathrm{H}} $, where $M_C$ is the molar mass of Carbon and $M_H$ is the molar mass of Hydrogen.

Use a \pythoninline{for}-loop or a \pythoninline{while}-loop to compute and print out the molar mass of the alkanes with two through nine Carbon atoms ($n \in [2, 9]$). The output should specify the chemical formula of the alkane as well as the molar mass. An example on how the formatted output should look like for $n = 2$ is given below.
\begin{lstlisting}
M(C2H6) =  30.069 g/mol
\end{lstlisting}
You can set the molar masses of the atoms to be $M_{\mathrm{C}} = 12.011 \ \mathrm{g/mol}$ and $M_{\mathrm{H}}  = 1.0079 \ \mathrm{g/mol}$

Hint: An output of a chemical formula like the one above can be constructed using a formatted string (f-string). If we have \py{n=2} and
\py{m=2*n+2}, the first part of the output above is given by the f-string \py{f'M(C{n:d}H{6:d} = ...'}.

Filename: $\texttt{alkane.py}$
\end{Problem}


\begin{Problem}{\textbf{Matrix elements}}

\begin{center}
\emph{This exercise involves no programming. \newline The answers should be written in a text file called $\texttt{matrix.txt}$}
\end{center}

\noindent Consider a two dimensional $3 \times 3$ matrix

\begin{equation*}
   A = \begin{pmatrix}
        a_{1 1} & a_{1 2} & a_{1 3} \\
        a_{2 1} & a_{2 2} & a_{2 3} \\
        a_{3 1} & a_{3 2} & a_{3 3}
    \end{pmatrix} .
\end{equation*}

In Python, the matrix $A$ can be represented as a nested list \pythoninline{A}, either as a list of rows or a list of columns. Find the indices \pythoninline{i, j} of the Python list \pythoninline{A} such that \pythoninline{A[i][j]} = $a_{1 1}$ and the indices \pythoninline{k, l}  such that \pythoninline{A[k][l]} = $a_{3 2}$ for the two following cases:

\paragraph{a)}
When $A$ is represented as a list of rows. This means that \pythoninline{A} contains three lists, where each list corresponds to a row in $A$.

\paragraph{b)}
When $A$ is represented as a list of columns. This means that each element in \pythoninline{A} contains a list with the elements of a column in $A$.


Filename: $\texttt{matrix.txt}$
\end{Problem}
