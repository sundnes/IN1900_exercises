%\section{Functions and branching}

\begin{Problem}{\textbf{Implement a function for population growth}}

\noindent Consider again the function
\begin{equation*}
N(t, k, B, C) = \frac{B}{1 + C \exp{-kt}}.
\end{equation*}
Implement $N$ as a python function \pythoninline{population(t, k, B, C)}
that returns the number of individuals in a population after a time $t$.

Use a \py{for} loop to write out a nicely formatted table of $t$ and $N$ values
for the time interval $t \in [0, 48]$ using the parameter values from Problem \ref{population_table}.

Filename: \texttt{pop\_func.py}
\end{Problem}


\begin{Problem}{\textbf{Sum of integers}}
%\addcontentsline{toc}{section}{Exercise 3.2: Sum of integers   - \texttt{sumint.py}}

\noindent We consider the sum $\sum_{i = 1}^n i=1+2+\dots+n$ of positive integers up to $n$.
It can be shown that the sum is equal to $\frac{n(n+1)}{2}$.

\paragraph{a)}
Write a function \pythoninline{sumint(n)} that returns the sum of all positive
integers up to n.

\paragraph{b)}
Write a function implementing $\frac{n(n+1)}{2}$.

\paragraph{c)}
Write test functions for both a) and b) testing for specific known values.

Filename: \texttt{sumint.py}
\end{Problem}


\begin{Problem}{\textbf{Implement the factorial}}
%\addcontentsline{toc}{section}{Exercise 3.3: Implement the factorial  - \texttt{factorial.py}}
\paragraph{a)} The factorial can be implemented by a so called recursive function
call. Use a recursive function call to implement a function \pythoninline{myfactorial(n)} that returns $n!$.

\paragraph{b)} Write a test function where you call the \pythoninline{myfactorial}
function and check the value of the returned object for one value of $n$ using
\pythoninline{math.factorial}.

Filename: $\texttt{factorial.py}$
\end{Problem}

\begin{Problem}{\textbf{Compute the area of an arbitrary triangle}} \label{triangle}
An arbitrary triangle can be described by the coordinates of its three vertices: $(x1, y1), (x2, y2), (x3, y3)$, numbered in a counterclockwise direction. The area of the triangle is given by the formula
\[
A = \frac{1}{2}|x_2y_3-x_3y_2-x_1y_3 + x_3y_1 + x_1y_2 - x_2y_1 |
\]
Write a function \pythoninline{triangle_area(vertices)} that returns
the area of a triangle whose vertices are specified by the argument \pythoninline{vertices},
which is a nested list of the vertex coordinates. Copy and paste the following
test function into your code, and call the test function to verify that
your \pythoninline{triangle_area} function works:
\begin{python}
def test_triangle_area():
    """
    Verify the area of a triangle with vertices
    (0,0), (1,0), and (0,2).
    """
    v1 = [0,0];  v2 = [1,0];  v3 = [0,2]
    vertices = [v1, v2, v3]
    expected = 1
    computed = triangle_area(vertices)
    tol = 1E-14
    success = abs(expected - computed) < tol
    msg = f"computed area={computed} != {expected}(expected)"
    assert success, msg
\end{python}

Filename: \texttt{triangle\_area.py}
\end{Problem}


\begin{Problem}{\textbf{Half-wave rectifier}} \label{prob34}
%\addcontentsline{toc}{section}{Exercise 3.4 - Half-wave rectifier - \texttt{half\_wave.py}}

\noindent In a half-wave rectifier the positive part of a signal passes, while the negative
part is blocked. Thus, for a signal passing through a half-wave rectifier,
the negative values are set to zero.
A sine signal that has passed through a half-wave rectifier will look as follows:
\begin{equation*}
f(x) = \begin{cases}
      \sin x & \mathrm{if} \sin x > 0 \\
      0 & \mathrm{if} \sin x \leq 0.
   \end{cases}
\end{equation*}
Implement $f(x)$ as a Python function \pythoninline{f(x)} and make a test function
for testing the implementation of \pythoninline{f(x)}. The test function should test
for at least two values of $x$, one that gives $\sin x <0$ and one where $\sin x >0$.

Filename: \texttt{half\_wave.py}
\end{Problem}

\begin{Problem}{\textbf{Primality checker}} \label{prime}

\noindent Recall that a prime number is a number greater than 1 that has exactly 2 divisors.
Said differently, a number greater than one is a prime if it is divisible by only
itself and one. A number that is not prime is called composite. Every number $n$
can be written as a unique product of primes (e.g. $12 = 2\cdot2\cdot 3 $),
this is called the prime factorization of $n$.

\paragraph{a)}
Make a function that takes a number $n$, and returns true if it's prime, and false
if it's not. Use the program to find all prime numbers up to 100.

\emph{Hint: You will only need to check divisibility for numbers up to and
including $\sqrt(n)$, because any greater divisor will imply that there
is a divisor less than this.}

\paragraph{b)}
Make a function that instead finds the prime factorization of the input
number. It should print ``prime'' and return nothing if the number is prime, and both
print and return the factorization if it's composite. Find the prime factorization
of 5525612.

\paragraph{c)}
Make test functions for the two functions above where you check for small values
of $n$.

\paragraph{d)}
Compare the runtime of the two functions with the number 33425626272.
Is the difference big? If so, why do you think one is faster than the other?
The following code returns the mean time it takes for your program to run once:
\begin{python}
import timeit
timeit.timeit('your_func(args)', \
    'from __main__ import your_func',number=1)
\end{python}

Filename: \texttt{prime.py}
\end{Problem}

\begin{Problem}{\textbf{Eulers totient function}}

\noindent Two numbers $n$ and $m$ are called relatively prime if they have no common divisors
except for 1. That is, no number greater than one should divide both numbers with
no residue.

\paragraph{a)}
Make a function that takes two numbers and returns true if they're relatively prime
and false if they're not.

\paragraph{b)}
Euler's totient function is defined as
\begin{equation*}
\phi(d)= \text{\#\{Numbers less than d which are
relatively prime to d\}.}
\end{equation*}
Implement Eulers totient function and print $\phi(d)$ for $d=10,50,100,200$.

\paragraph{c)}
Make a test function for both a) and b).

Filename: \texttt{euler.py}
\end{Problem}

\begin{Problem}{\textbf{Simple Statistical Functions}}

\noindent In this problem you will implement two statistical functions and test them by comparing the results with statistical functions from the  \pythoninline{numpy} module. We will trust that the functions from the \pythoninline{numpy} module are correct, and will use them as benchmark values in the test functions. When you import the \pythoninline{numpy} module you should follow the convention of renaming it \pythoninline{np}, as shown below.
\begin{python}
import numpy as np
\end{python}

\paragraph{a)}
The mean of a set of numbers ${x_1, x_2, x_3, ..., x_N}$ is defined as
\begin{equation*}
    \overline{x} = \frac{1}{N} \sum_{i=1}^{N} x_i ,
\end{equation*}
where $N$ is the size of the set.
Implement a function  \pythoninline{mean(x_list)} that returns the mean value of a list of numbers.

\paragraph{b)} Make a test function \pythoninline{test_mean()} that tests the function from a). Compare the returned value with the result from \pythoninline{numpy.mean}. (Such that

\noindent
\pythoninline{expected = np.mean(x_test_values)}).


\paragraph{c)} The standard deviation of a set of numbers ${x_1, x_2, x_3, ..., x_N}$ is defined as
\begin{equation*}
    s_N = \sqrt{ \frac{1}{N}
        \sum_{i=1}^{N} \left(x_i - \overline{x} \right)^2} .
\end{equation*}
Implement a function \pythoninline{standard_deviation(x_list)} which returns the standard deviation of a list of numbers. The mean value of the list will be necessary to calculate the relative deviation. Obtain the mean value inside the  \pythoninline{standard_deviation} function by calling the function you made in a).

\paragraph{d)} Make a test function \pythoninline{test_standard_deviation()} that tests the function from c). Compare the returned value with the result from
\pythoninline{numpy.std}. (Such that \pythoninline{expected = np.std(x_test_values)}).

\paragraph{} You may use the list below as an example for your test functions.
\begin{python}
x_test_values = [0.699, 0.703, 0.698, 0.688, 0.701]
\end{python}

Filename: $\texttt{stat.py}$
\end{Problem}

\begin{Problem}{\textbf{Münchhausen Numbers}}

\noindent A Münchhausen number is a number such that the sum of every digit to the power
of itself equals the original number. E.g. $1^1 = 1$ is a Münchhausen number, and
 $5^5 + 3^3 + 2^2 = 3156\neq 532$, so 532 is not.

Make a function that checks if a number is Münchhausen. Find a Münchhausen number different from one.

\emph{Hint: There is only one such number different from 1 and also under one million}

Filename: $\texttt{m\_numbers.py}$
\end{Problem}
