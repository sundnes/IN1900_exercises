%\section{Random numbers and simple games}

\begin{Problem}{\textbf{Throw a die}}
%\addcontentsline{toc}{section}{Exercise 8.1: Throw a die - \texttt{die.py}}

\noindent Compute the probability of getting a $6$ when throwing a die. Write a program
that throws a die $N$ times and count how many times the die shows $6$, let this
number be $M$. Then compute the probability of getting a $6$ when throwing a die
as $M/N$.

Filename: $\texttt{die.py}$
\end{Problem}

\begin{Problem}{\textbf{Telephone number}}
%\addcontentsline{toc}{section}{Exercise 8.2:  - \texttt{telephone.py}}

\noindent A Norwegian telephone  number consists of eight digits. We assume that all digits
from $0$ to $9$ are equally probable in every place of the telephone number.
Make a program that finds the probability of having a telephone number where the
digit $1$ appears at least four times.

Filename: $\texttt{telephone.py}$
\end{Problem}

\begin{Problem}{\textbf{Coin-flip game}}
%\addcontentsline{toc}{section}{Exercise 8.3: Coin-flip game - \texttt{coin.py}}

\noindent Two persons are playing a simple coin-flip game. They flip a coin in turn,
and whoever first gets a heads wins the game. Make a program to model $100$ such
games. Estimate the probability for the first person to flip to win the game.

Filename: $\texttt{coin.py}$
\end{Problem}

\begin{Problem}{\textbf{Birthday probability}}

\noindent Make a function that generates a string of random integers between 0 and 9.
Estimate the probability that your birthday is contained in a string of random
numbers of length 1000. Let the format of the date be on the form \pythoninline{ddmmyy}.
Print the estimates in \%.

Filename: \texttt{birthday\_prob.py}
\end{Problem}
\newpage

\begin{Problem}{\textbf{Approximate $\pi$ by throwing darts}}
%\addcontentsline{toc}{section}{Exercise 8.4: Approximate $\pi$ by throwing darts - \texttt{approximate\_pi.py}}

\noindent You are throwing darts at a square shaped target with an inscribed circle. Let
the length of the sides of the square be $2$, which means that the circle has
radius $1$. Assume that you throw the darts such that the darts gets uniformly
distributed on the target. Then, the number of darts which hits the target inside
the circle divided by the total number of darts that hits the target is
approximately the area of the circle divided by the total area of the target.
This approximation gets more accurate the more darts you throw.
\begin{equation*}
\frac{\mathrm{number\; of\; darts\; inside\; circle}}
{\mathrm{number\; of\; darts\; that\; hits\; target}}
\approx \frac{\mathrm{area\; of\; circle}}{\mathrm{area\; of\; target}}
= \frac{\pi}{4}.
\end{equation*}
Thus, $\pi$ can be approximated by
\begin{equation*}
\pi \approx 4 \frac{\mathrm{number\; of\; darts\; inside\; circle}}
{\mathrm{number\; of\; darts\; that\; hits\; target}}.
\end{equation*}
Write a program that throws $M$ darts uniformly on the target. Then approximate
$\pi$. Read $M$ from the command line.

Filename: $\texttt{approximate\_pi.py}$
\end{Problem}

\begin{Problem}{\textbf{Wheel of fortune}}
%\addcontentsline{toc}{section}{Exercise 8.5: Wheel of fortune - \texttt{wheel\_of\_fortune.py}}

\noindent At an amusement park they have a wheel of fortune where you can win $2$kg of
chocolate. You get to choose one number between $1$ and $20$ for $20$NOK. Assume that you
play on the same number until you win.

\paragraph{a)}
Write a program that finds the average number of times you have to play before
you win and check if you earn or loose money, compared to buying 2kg of chocolate in the store.

\paragraph{b)}
Modify your program so that every time you lose you move one place to the right,
i.e. you increase $n$ by one. If you are at $n = 20$ you go back to $n = 1$.
Does this make any difference to the result?

Filename: \texttt{wheel\_of\_fortune.py}
\end{Problem}
