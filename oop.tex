%\section{Object oritented programming}
The topic of this chapter is object-oriented programming (OOP), which is
covered in Chapter 9 of the book by Sundnes and the book by Langtangen.


\begin{Problem}{\textbf{Implement Newtons method}} \label{prob91}

\paragraph{a)}
Make a subclass \pythoninline{Function} of the class \pythoninline{Diff} in problem \ref{prob74}
that takes a function $f$ as
an initial variable. It should contain a method such that the following code
is compatible with your program and prints the value of $f$ at 2.
\begin{python}
def f(x):
    return x**2
func = Function(f)
print(f(2))
\end{python}

\paragraph{b)}
We would like the class to give estimated values for roots of $f$. That is,
points such that $f(x)=0$. To do this we implement Newton's formula. It is given
recursively as
\begin{equation*}
    x_{n+1}=x_n-\frac{f(x_n)}{f'(x_n)},
\end{equation*}
where we give a starting point $x_0$. In some cases(not all) $x_n$ will approach
a root of $f$. Implement this in a method \pythoninline{approx_root} that takes a
starting point and a bound $\epsilon<1$ as arguments and approximates $x_n$ such that
$f(x_n)<\epsilon$.

\emph{Hint: Implement a simple convergence test. Check that $f(x_n)<1$ after $100$ iterations. If
not terminate the loop and inform the user that there is no convergence for that
starting point. It is still a possibility for convergence, but unlikely.}

\paragraph{c)}
Test the program with the function $f(x)=x^2-1$ and starting value 5 with bounds
$10^{-i}$, for $i=1,2,3,4,5,6$. Print the approximated value for $x$, $f(x)$ and the
bound in a table format. Try to run the program with starting value 0. What happens,
can you see why?

Filename: \texttt{newton.py}
\end{Problem}

\newpage
\begin{Problem}{\textbf{Implement Polynomials as a Class}} \label{prob92}

\paragraph{a)}
Make a class \pythoninline{Quadratic} that implements second order polynomials. An
object of \pythoninline{Quadratic} should be initialised with a list containing the
coefficients. Add a \pythoninline{__call__(self,x)} method that evaluates the
polynomial at a point \pythoninline{x}, and a \pythoninline{__str__(self)} method
for printing the polynomial.
\\
Create an instance of \pythoninline{Quadratic} with coefficients $(1, 3, 2)$. Print it and
evaluate it at $x=1$, $x=2$ to make sure it works.

\paragraph{b)}
Make a subclass \pythoninline{Cubic} of the class \pythoninline{Quadratic} that implements third order
polynomials. You should make use of inheritance to extend the class
\pythoninline{Quadratic} you made in the previous exercise. Implement a method
\pythoninline{derivative} for \pythoninline{Cubic}. The method should return an object
of type \pythoninline{Quadratic} that corresponds to the derivative.
\\
Create an instance of \pythoninline{Cubic} with coefficients $(1, 3, 2, 4)$. Print it
and evaluate it at $x=1$, $x=2$. Also call \pythoninline{derivative} and print the result.

Filename: \texttt{polynomial.py}
\end{Problem}

\begin{Problem}{\textbf{Vectors}} \label{prob93}

\paragraph{a)}

Make a class Vector2D that implements 2D-vectors(two components). Add \pythoninline{__add__} and \pythoninline{__sub__} methods so you can add and subtract vectors. Remember that vectors add and subtract element-wise, e.i.:
\begin{equation*}
    (1,2) + (4,5) = (5,7)
\end{equation*}

\paragraph{b)}
For two vectors $\Vec{a} = (a_1, a_2)$ and $\Vec{b} = (b_1, b_2)$, the dot product is defined as
\begin{equation*}
    \Vec{a} \cdot \Vec{b} = a_1 b_1 + a_2 b_2
\end{equation*}
\\
Implement a method \pythoninline{dot} that calculates the dot product of two vectors.
\\
Make two vectors $\Vec{v}=(1,2)$ and $\Vec{w}=(-2,5)$ from Vector2D. Print $\Vec{v}$, $\Vec{w}$, $\Vec{v}+\Vec{w}$, $\Vec{v}-\Vec{w}$ and $\Vec{v}\cdot \Vec{w}$

\paragraph{c)}
Make a subclass Vector3D of the class Vector2D, where Vector3D is extended with an additional coordinate. Make sure all the methods are updated to work with three coordinates. The dot product for 3D vectors extends as one would expect:

\begin{equation*}
    \Vec{a} \cdot \Vec{b} = a_1 b_1 + a_2 b_2 + a_3 b_3
\end{equation*}
Use inheritance to reuse old code as much as possible(especially for the dot product method).
\\
Make two vectors $\Vec{v}=(1,2,4)$ and $\Vec{w}=(-2,5,1)$ from Vector3D. Print $\Vec{v}$, $\Vec{w}$, $\Vec{v}+\Vec{w}$, $\Vec{v}-\Vec{w}$ and $\Vec{v}\cdot \Vec{w}$

\paragraph{d)}
There is a common vector operation that is defined for 3D vectors, but not for 2D vectors. This is the cross product. It differs from the dot product in that the result is not a number, but a new vector:
\begin{equation*}
    \Vec{c} = \Vec{a} \times \Vec{b}
\end{equation*}
where $\Vec{c} = (c_1, c_2, c_3)$. The cross product is define as
\begin{align*}
    c_1 = a_2 b_3 - a_3 b_2\\
    c_2 = a_3 b_1 - a_1 b_3\\
    c_3 = a_1 b_2 - a_2 b_1\\
\end{align*}
Implement a method \pythoninline{cross} for Vector3D only that calculates the cross product. \\
Make two vectors $\Vec{v}=(2,0,0)$, $\Vec{w}=(0,2,0)$ from Vector3D. Print $v\times w$.
\\
Filename: \texttt{vector.py}
\end{Problem}

\begin{Problem}{\textbf{Inheritance}} \label{prob94}
\\
In this exercise we will investigate how python handles inheritance by making some intuitive classes.

\paragraph{a)}
Begin by making a class \pythoninline{Mammal}. Add a
method \pythoninline{info(self)} that returns a string stating
something that is common to all mammals, for instance "I have hair on my body".
Also add a method \pythoninline{identity_mammal(self)} that prints (not returns)
"I am a mammal".

\paragraph{b)}
Make a subclass Primate of the class Mammal. Add a method \pythoninline{info(self)}
that returns the same as \pythoninline{info(self)} for Mammal, in addition to
something specific for Primates, for instance "I have a large brain". Use inheritance
to include the general string from \pythoninline{Mammal}. Do not copy-paste it
into the Primate class. Also add a method
\pythoninline{identity_primate(self)} equivalent to \pythoninline{identity_mammal(self)}.

\paragraph{c)}
Now make two subclasses Human and Ape from Primate. Update the \pythoninline{info(self)}
method in the same manner for both Human and Ape, and also add their respective
identity methos.

Make an instance John of the class Human, and an object Julius of the class Ape.
Try calling \pythoninline{info()}, \pythoninline{identity_mammal()},
\pythoninline{identity_primate()},  \pythoninline{identity_human()} and \pythoninline{identity_ape()} for both John and Julius. Does it work as you expect? Some of these calls are meant to cause an error.
\paragraph{d)}
Python is able to check if an object is of a particular class with the function \pythoninline{isinstance}.
\pythoninline{isinstance(object_name, class_name)} returns \pythoninline{True} if the object \pythoninline{object_name} is of class \pythoninline{class_name}. An example could be
\pythoninline{isinstance(John, Primate)}, which returns \pythoninline{True} if John is of the class Primate.


Use \pythoninline{isinstance} to check if John is Mammal, Primate, Human and Ape. Do the same for Julius.

Filename: \texttt{inheritance.py}
\end{Problem}
